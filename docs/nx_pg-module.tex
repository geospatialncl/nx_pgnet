%
% API Documentation for NetworkX - PostGIS Python interface
% Module nx_pg
%
% Generated by epydoc 3.0.1
% [Wed Aug  1 10:29:57 2012]
%

%%%%%%%%%%%%%%%%%%%%%%%%%%%%%%%%%%%%%%%%%%%%%%%%%%%%%%%%%%%%%%%%%%%%%%%%%%%
%%                          Module Description                           %%
%%%%%%%%%%%%%%%%%%%%%%%%%%%%%%%%%%%%%%%%%%%%%%%%%%%%%%%%%%%%%%%%%%%%%%%%%%%

    \index{nx\_pg \textit{(module)}|(}
\section{Module nx\_pg}

    \label{nx_pg}
nx\_pg.py - Read/write support for PostGIS tables in NetworkX

\textbf{Introduction}

NetworkX is a python library for graph analysis. Using edge and node 
attribution it can be used for spatial network analysis (with geography 
stored as node/edge attributes). This module supports the use of NetworkX 
for the development of network analysis of spatial networks stored in a 
PostGIS spatial database by acting as an interface to node and edge tables 
which contain graph nodes and edges.

\textbf{Notes}

\textit{Node support}

Note that nodes are defined by the edges at network creation and the 
reading of node tables independently is not currently supported (because 
without defining a primary key/foreign key relationship this can easily 
break the network). This issue is solved in nx\_pgnet which should be used 
for proper storage of networks in PostGIS tables. To read/write PostGIS 
networks (as defined by a network schema use) the nx\_pgnet module.

\textit{Output tables}

For each network written two tables are created: edges and nodes. This 
representation is similar to that of the nx\_shp module (for 
reading/writing network shapefiles).

\textit{Coordinate system support}

nx\_pg has no support for defining a coordinate system of the output 
tables. Geometry is written without an SRS value, when viewing using a GIS 
you must specify the correct coordinate system for the network, nx\_pgnet 
has coordinate systems support for network tables.

\textit{Graph/Network terms}

Note that the terms 'graph' and 'network' are used interchangeably within 
the software and documentation. To some extent a 'graph' refers to a 
topological object (often in memory) with none or limited attribution, 
whilst a 'network' refers to a graph object with attribution of edges and 
nodes and with geography defined, although this is not always the case.

\textbf{Module structure}

The module has two key functions:

\begin{itemize}
\setlength{\parskip}{0.6ex}
  \item read\_pg: Function to create NetworkX graph instance from PostGIS 
    table(s) representing edge and node objects.

  \item write\_pg: Function to create PostGIS tables (edge and node) from 
    NetworkX graph instance.

  \item Other functions support the read and write operations.

\end{itemize}

\textbf{Database connections}

Connections to PostGIS are created using the OGR simple feature library and
are passed to the read() and write() classes. See 
\href{http://www.gdal.org/ogr}{\textit{http://www.gdal.org/ogr}}

Connections are mutually exclusive between read\_pg and write\_pg, although
you can of course read and write to the same database. You must pass a 
valid connection to the read or write classes for the module to work.

To create a connection using the OGR python bindings to a database on 
localhost:

\begin{alltt}
\pysrcprompt{{\textgreater}{\textgreater}{\textgreater} }\pysrckeyword{import} osgeo.ogr \pysrckeyword{as} ogr
\pysrcprompt{{\textgreater}{\textgreater}{\textgreater} }conn = ogr.Open("PG: host=\pysrcstring{'127.0.0.1'} dbname=\pysrcstring{'database'} user=\pysrcstring{'postgres'}
\pysrcprompt{{\textgreater}{\textgreater}{\textgreater} }                password=\pysrcstring{'password'}")\end{alltt}
\textbf{Examples}

The following are examples of high level read and write network operations.
For more detailed information see method documentation below.

Reading a network from PostGIS table of LINESTRINGS representing edges:

\begin{alltt}
\pysrcprompt{{\textgreater}{\textgreater}{\textgreater} }\pysrckeyword{import} nx\_pg
\pysrcprompt{{\textgreater}{\textgreater}{\textgreater} }\pysrckeyword{import} osgeo.ogr \pysrckeyword{as} ogr\end{alltt}
\begin{alltt}
\pysrcprompt{{\textgreater}{\textgreater}{\textgreater} }\pysrccomment{\# Create a connection}
\pysrcprompt{{\textgreater}{\textgreater}{\textgreater} }conn = ogr.Open("PG: host=\pysrcstring{'127.0.0.1'} dbname=\pysrcstring{'database'} user=\pysrcstring{'postgres'}
\pysrcprompt{{\textgreater}{\textgreater}{\textgreater} }                password=\pysrcstring{'password'}")\end{alltt}
\begin{alltt}
\pysrcprompt{{\textgreater}{\textgreater}{\textgreater} }\pysrccomment{\# Read a network}
\pysrcprompt{{\textgreater}{\textgreater}{\textgreater} }\pysrccomment{\# Note 'my\_network' is the name of the network stored in the 'Graphs' table}\end{alltt}
\begin{alltt}
\pysrcprompt{{\textgreater}{\textgreater}{\textgreater} }network = nx\_pg.read\_pg(conn, \pysrcstring{'edges\_tablename'})\end{alltt}
\textit{Adding node attributes}

Nodes are created automatically at the start/end of a line, or where there 
is a break in a line. A user can add node attributes from a table of point 
geometries which represent these locations. To add attributes to nodes use 
the nodes\_tablename option in the read function:

\begin{alltt}
\pysrcprompt{{\textgreater}{\textgreater}{\textgreater} }network = nx\_pg.read\_pg(conn, \pysrcstring{'edge\_tablename'}, \pysrcstring{'node\_tablename'})\end{alltt}
This will add attributes from the node table to nodes in the network where 
a network node geometry is equal to a point geometry in the specified node 
table.

Note that this will not add all points in the nodes table if not all points
match the geometry of created nodes.

Also note that if two points share the same geometry as a node, only one of
the point attributes will be added (whichever occurs later in the data).

\textit{Writing a NetworkX graph instance to a PostGIS schema}:

Write the network to the same database but under a different name. Note if 
'overwrite=True' then an existing network in the database of the same name 
will be overwritten.

\begin{alltt}
\pysrcprompt{{\textgreater}{\textgreater}{\textgreater} }nx\_pg.write\_pg(conn, network, \pysrcstring{'new\_network, overwrite=False'})\end{alltt}
\textbf{Dependencies}

\begin{itemize}
\setlength{\parskip}{0.6ex}
  \item Python 2.6 or later

  \item NetworkX 1.6 or later

  \item OGR 1.8.0 or later

\end{itemize}

\textbf{Copyright}

Tomas Holderness \& Newcastle University

Developed by Tom Holderness at Newcastle University School of Civil 
Engineering and Geosciences, Geoinfomatics group:

\textbf{Acknowledgement}

Acknowledgement must be made to the nx\_shp developers as much of the 
functionality of this module is the same.

\textbf{License}

This software is released under a BSD style license which must be 
compatible with the NetworkX license because of similarities with NetworkX 
source code.:

See LICENSE.TXT or type nx\_pg.license() for full details.

\textbf{Credits}

Tomas Holderness, David Alderson, Alistair Ford, Stuart Barr and Craig 
Robson.

\textbf{Contact}

tom.holderness@ncl.ac.uk

www.staff.ncl.ac.uk/tom.holderness

\textbf{Version:} 0.9.1



\textbf{Author:} Tom Holderness




%%%%%%%%%%%%%%%%%%%%%%%%%%%%%%%%%%%%%%%%%%%%%%%%%%%%%%%%%%%%%%%%%%%%%%%%%%%
%%                               Functions                               %%
%%%%%%%%%%%%%%%%%%%%%%%%%%%%%%%%%%%%%%%%%%%%%%%%%%%%%%%%%%%%%%%%%%%%%%%%%%%

  \subsection{Functions}

    \label{nx_pg:getfieldinfo}
    \index{nx\_pg \textit{(module)}!nx\_pg.getfieldinfo \textit{(function)}}

    \vspace{0.5ex}

\hspace{.8\funcindent}\begin{boxedminipage}{\funcwidth}

    \raggedright \textbf{getfieldinfo}(\textit{lyr}, \textit{feature}, \textit{flds})

    \vspace{-1.5ex}

    \rule{\textwidth}{0.5\fboxrule}
\setlength{\parskip}{2ex}
    Get information about fields from a table (as OGR feature).

\setlength{\parskip}{1ex}
    \end{boxedminipage}

    \label{nx_pg:read_pg}
    \index{nx\_pg \textit{(module)}!nx\_pg.read\_pg \textit{(function)}}

    \vspace{0.5ex}

\hspace{.8\funcindent}\begin{boxedminipage}{\funcwidth}

    \raggedright \textbf{read\_pg}(\textit{conn}, \textit{edgetable}, \textit{nodetable}={\tt None}, \textit{directed}={\tt False})

    \vspace{-1.5ex}

    \rule{\textwidth}{0.5\fboxrule}
\setlength{\parskip}{2ex}
    Read a network from PostGIS table of line geometry.

    Optionally takes a table of points and where point geometry is equal to
    that of nodes created, point attributes will be added to nodes.

    Returns instance of networkx.Graph().

\setlength{\parskip}{1ex}
    \end{boxedminipage}

    \label{nx_pg:netgeometry}
    \index{nx\_pg \textit{(module)}!nx\_pg.netgeometry \textit{(function)}}

    \vspace{0.5ex}

\hspace{.8\funcindent}\begin{boxedminipage}{\funcwidth}

    \raggedright \textbf{netgeometry}(\textit{key}, \textit{data})

    \vspace{-1.5ex}

    \rule{\textwidth}{0.5\fboxrule}
\setlength{\parskip}{2ex}
    Create OGR geometry from a NetworkX Graph using Wkb/Wkt attributes.

\setlength{\parskip}{1ex}
    \end{boxedminipage}

    \label{nx_pg:create_feature}
    \index{nx\_pg \textit{(module)}!nx\_pg.create\_feature \textit{(function)}}

    \vspace{0.5ex}

\hspace{.8\funcindent}\begin{boxedminipage}{\funcwidth}

    \raggedright \textbf{create\_feature}(\textit{geometry}, \textit{lyr}, \textit{attributes}={\tt None})

    \vspace{-1.5ex}

    \rule{\textwidth}{0.5\fboxrule}
\setlength{\parskip}{2ex}
    Wrapper for OGR CreateFeature function.

    Creates a feature in the specified table with geometry and attributes.

\setlength{\parskip}{1ex}
    \end{boxedminipage}

    \label{nx_pg:write_pg}
    \index{nx\_pg \textit{(module)}!nx\_pg.write\_pg \textit{(function)}}

    \vspace{0.5ex}

\hspace{.8\funcindent}\begin{boxedminipage}{\funcwidth}

    \raggedright \textbf{write\_pg}(\textit{conn}, \textit{network}, \textit{tablename\_prefix}, \textit{overwrite}={\tt False})

    \vspace{-1.5ex}

    \rule{\textwidth}{0.5\fboxrule}
\setlength{\parskip}{2ex}
    Write NetworkX instance to PostGIS edge and node tables.

\setlength{\parskip}{1ex}
    \end{boxedminipage}


%%%%%%%%%%%%%%%%%%%%%%%%%%%%%%%%%%%%%%%%%%%%%%%%%%%%%%%%%%%%%%%%%%%%%%%%%%%
%%                               Variables                               %%
%%%%%%%%%%%%%%%%%%%%%%%%%%%%%%%%%%%%%%%%%%%%%%%%%%%%%%%%%%%%%%%%%%%%%%%%%%%

  \subsection{Variables}

    \vspace{-1cm}
\hspace{\varindent}\begin{longtable}{|p{\varnamewidth}|p{\vardescrwidth}|l}
\cline{1-2}
\cline{1-2} \centering \textbf{Name} & \centering \textbf{Description}& \\
\cline{1-2}
\endhead\cline{1-2}\multicolumn{3}{r}{\small\textit{continued on next page}}\\\endfoot\cline{1-2}
\endlastfoot\raggedright \_\-\_\-c\-r\-e\-a\-t\-e\-d\-\_\-\_\- & \raggedright \textbf{Value:} 
{\tt \texttt{'}\texttt{Thu Jan 19 15:55:13 2012}\texttt{'}}&\\
\cline{1-2}
\raggedright \_\-\_\-y\-e\-a\-r\-\_\-\_\- & \raggedright \textbf{Value:} 
{\tt \texttt{'}\texttt{2011}\texttt{'}}&\\
\cline{1-2}
\raggedright \_\-\_\-p\-a\-c\-k\-a\-g\-e\-\_\-\_\- & \raggedright \textbf{Value:} 
{\tt None}&\\
\cline{1-2}
\end{longtable}


%%%%%%%%%%%%%%%%%%%%%%%%%%%%%%%%%%%%%%%%%%%%%%%%%%%%%%%%%%%%%%%%%%%%%%%%%%%
%%                           Class Description                           %%
%%%%%%%%%%%%%%%%%%%%%%%%%%%%%%%%%%%%%%%%%%%%%%%%%%%%%%%%%%%%%%%%%%%%%%%%%%%

    \index{nx\_pg \textit{(module)}!nx\_pg.Error \textit{(class)}|(}
\subsection{Class Error}

    \label{nx_pg:Error}
\begin{tabular}{cccccccccc}
% Line for object, linespec=[False, False, False]
\multicolumn{2}{r}{\settowidth{\BCL}{object}\multirow{2}{\BCL}{object}}
&&
&&
&&
  \\\cline{3-3}
  &&\multicolumn{1}{c|}{}
&&
&&
&&
  \\
% Line for exceptions.BaseException, linespec=[False, False]
\multicolumn{4}{r}{\settowidth{\BCL}{exceptions.BaseException}\multirow{2}{\BCL}{exceptions.BaseException}}
&&
&&
  \\\cline{5-5}
  &&&&\multicolumn{1}{c|}{}
&&
&&
  \\
% Line for exceptions.Exception, linespec=[False]
\multicolumn{6}{r}{\settowidth{\BCL}{exceptions.Exception}\multirow{2}{\BCL}{exceptions.Exception}}
&&
  \\\cline{7-7}
  &&&&&&\multicolumn{1}{c|}{}
&&
  \\
&&&&&&\multicolumn{2}{l}{\textbf{nx\_pg.Error}}
\end{tabular}

Class to handle network IO errors.


%%%%%%%%%%%%%%%%%%%%%%%%%%%%%%%%%%%%%%%%%%%%%%%%%%%%%%%%%%%%%%%%%%%%%%%%%%%
%%                                Methods                                %%
%%%%%%%%%%%%%%%%%%%%%%%%%%%%%%%%%%%%%%%%%%%%%%%%%%%%%%%%%%%%%%%%%%%%%%%%%%%

  \subsubsection{Methods}

    \vspace{0.5ex}

\hspace{.8\funcindent}\begin{boxedminipage}{\funcwidth}

    \raggedright \textbf{\_\_init\_\_}(\textit{self}, \textit{value})

\setlength{\parskip}{2ex}
    x.\_\_init\_\_(...) initializes x; see x.\_\_class\_\_.\_\_doc\_\_ for 
    signature

\setlength{\parskip}{1ex}
      Overrides: object.\_\_init\_\_ 	extit{(inherited documentation)}

    \end{boxedminipage}

    \vspace{0.5ex}

\hspace{.8\funcindent}\begin{boxedminipage}{\funcwidth}

    \raggedright \textbf{\_\_str\_\_}(\textit{self})

\setlength{\parskip}{2ex}
    str(x)

\setlength{\parskip}{1ex}
      Overrides: object.\_\_str\_\_ 	extit{(inherited documentation)}

    \end{boxedminipage}


\large{\textbf{\textit{Inherited from exceptions.Exception}}}

\begin{quote}
\_\_new\_\_()
\end{quote}

\large{\textbf{\textit{Inherited from exceptions.BaseException}}}

\begin{quote}
\_\_delattr\_\_(), \_\_getattribute\_\_(), \_\_getitem\_\_(), \_\_getslice\_\_(), \_\_reduce\_\_(), \_\_repr\_\_(), \_\_setattr\_\_(), \_\_setstate\_\_(), \_\_unicode\_\_()
\end{quote}

\large{\textbf{\textit{Inherited from object}}}

\begin{quote}
\_\_format\_\_(), \_\_hash\_\_(), \_\_reduce\_ex\_\_(), \_\_sizeof\_\_(), \_\_subclasshook\_\_()
\end{quote}

%%%%%%%%%%%%%%%%%%%%%%%%%%%%%%%%%%%%%%%%%%%%%%%%%%%%%%%%%%%%%%%%%%%%%%%%%%%
%%                              Properties                               %%
%%%%%%%%%%%%%%%%%%%%%%%%%%%%%%%%%%%%%%%%%%%%%%%%%%%%%%%%%%%%%%%%%%%%%%%%%%%

  \subsubsection{Properties}

    \vspace{-1cm}
\hspace{\varindent}\begin{longtable}{|p{\varnamewidth}|p{\vardescrwidth}|l}
\cline{1-2}
\cline{1-2} \centering \textbf{Name} & \centering \textbf{Description}& \\
\cline{1-2}
\endhead\cline{1-2}\multicolumn{3}{r}{\small\textit{continued on next page}}\\\endfoot\cline{1-2}
\endlastfoot\multicolumn{2}{|l|}{\textit{Inherited from exceptions.BaseException}}\\
\multicolumn{2}{|p{\varwidth}|}{\raggedright args, message}\\
\cline{1-2}
\multicolumn{2}{|l|}{\textit{Inherited from object}}\\
\multicolumn{2}{|p{\varwidth}|}{\raggedright \_\_class\_\_}\\
\cline{1-2}
\end{longtable}

    \index{nx\_pg \textit{(module)}!nx\_pg.Error \textit{(class)}|)}
    \index{nx\_pg \textit{(module)}|)}
