%
% API Documentation for NetworkX - PostGIS Python interface
% Module nx_pgnet
%
% Generated by epydoc 3.0.1
% [Wed Aug  1 10:29:57 2012]
%

%%%%%%%%%%%%%%%%%%%%%%%%%%%%%%%%%%%%%%%%%%%%%%%%%%%%%%%%%%%%%%%%%%%%%%%%%%%
%%                          Module Description                           %%
%%%%%%%%%%%%%%%%%%%%%%%%%%%%%%%%%%%%%%%%%%%%%%%%%%%%%%%%%%%%%%%%%%%%%%%%%%%

    \index{nx\_pgnet \textit{(module)}|(}
\section{Module nx\_pgnet}

    \label{nx_pgnet}
nx\_pgnet - Read/write support for PostGIS network schema in NetworkX.

\textbf{Introduction}

nx\_pgnet is module for reading and writing NetworkX graphs to and from 
PostGIS tables as specified by the Newcastle University PostGIS network 
schema.

Note that the terms 'graph' and 'network' are used interchangeably within 
the software and documentation. To some extent a 'graph' refers to a 
topological object (often in memory) with none or limited attribution, 
whilst a 'network' refers to a graph object with attribution of edges and 
nodes and with geography defined, although this is not always the case.

nx\_pgnet operations

read: PostGIS (network schema) --{\textgreater} NetworkX

write: PostGIS (network schema) {\textless}-- NetworkX

\textbf{Description}

NetworkX is a python library for graph analysis. Using edge and node 
attribution it can be used for spatial network analysis (with geography 
stored as node/edge attributes). This module supports the use of NetworkX 
for the development of network analysis of spatial networks stored in a 
PostGIS spatial database by acting as an interface to a predefined table 
structure (the schema) in a PostGIS database from NetworkX Graph classes.

\textbf{PostGIS Schema}

This module assumes that the required PostGIS network schema is available 
within the target/source PostGIS database. The schema allows for a 
collection of tables to represent a spatial network, storing network 
geography and attributes. The definition of the schema and the associated 
scripts for network creation are outside the scope of this documentation 
however the schema can be briefly described as the following tables:

\begin{itemize}
\setlength{\parskip}{0.6ex}
  \item Graphs: Holds a reference to all networks in the database, referencing 
    Edge, Edge\_Geometry and Node tables.

  \item Edges: Holds a representation of a network edge by storing source and 
    destination nodes and edge attributes. Contains foreign keys to graph 
    and edge geometry

  \item Edge\_Geometry: Holds geometry (PostGIS binary 
    LINESTRING/MULTILINESTRING representation). Edge geometry is stored 
    separately to edges for storage/retrieval performance where more than 
    one edge share the same geometry.

  \item Interdependency: Holds interdependencies between networks.

  \item Interdependency\_Edges: Holds interdependency geometry.

\end{itemize}

\textbf{Module structure    }

The module is split into three key classes:

\begin{itemize}
\setlength{\parskip}{0.6ex}
  \item read: Contains methods to read data from PostGIS network schema to a 
    NetworkX graph.

  \item write: Contains methods to write a NetworkX graph to PostGIS network 
    schema tables.

  \item nisql: Contains methods which act as a wrapper to the special PostGIS 
    network schema functions.

  \item errors: Class containing error catching, reporting and logging methods.

\end{itemize}

Detailed documentation for each class can be found below contained in class
document strings. The highest level functions for reading and writing data 
are:

Read:

\begin{alltt}
\pysrcprompt{{\textgreater}{\textgreater}{\textgreater} }nx\_pgnet.read().pgnet()
\pysrcprompt{{\textgreater}{\textgreater}{\textgreater} }\pysrccomment{\# Reads a network from PostGIS network schema into a NetworkX graph instance.}\end{alltt}
Write:

\begin{alltt}
\pysrcprompt{{\textgreater}{\textgreater}{\textgreater} }nx\_pgnet.write().pgnet()
\pysrcprompt{{\textgreater}{\textgreater}{\textgreater} }\pysrccomment{\# Writes a NetworkX graph instance to PostGIS network schema tables.}\end{alltt}
\textbf{Database connections}

Connections to PostGIS are created using the OGR simple feature library and
are passed to the read() and write() classes. See http://www.gdal.org/ogr

Connections are mutually exclusive between read() and write() and are 
contained within each class (i.e. all methods within those classes inherit 
the : connection), although you can of course read and write to the same 
database. You must pass a valid connection to the read or write classes for
the module to work.

To create a connection using the OGR python bindings to a database on 
localhost:

\begin{alltt}
\pysrcprompt{{\textgreater}{\textgreater}{\textgreater} }\pysrckeyword{import} osgeo.ogr \pysrckeyword{as} ogr
\pysrcprompt{{\textgreater}{\textgreater}{\textgreater} }conn = ogr.Open("PG: host=\pysrcstring{'127.0.0.1'} dbname=\pysrcstring{'database'} user=\pysrcstring{'postgres'}
\pysrcprompt{{\textgreater}{\textgreater}{\textgreater} }                password=\pysrcstring{'password'}")\end{alltt}
\textbf{Examples}

The following are examples of read and write network operations. For more 
detailed information see method documentation below.

Reading a network from PostGIS schema to a NetworkX graph instance:

\begin{alltt}
\pysrcprompt{{\textgreater}{\textgreater}{\textgreater} }\pysrckeyword{import} nx\_pgnet
\pysrcprompt{{\textgreater}{\textgreater}{\textgreater} }\pysrckeyword{import} osgeo.ogr \pysrckeyword{as} ogr\end{alltt}
\begin{alltt}
\pysrcprompt{{\textgreater}{\textgreater}{\textgreater} }\pysrccomment{\# Create a connection}
\pysrcprompt{{\textgreater}{\textgreater}{\textgreater} }conn = ogr.Open("PG: host=\pysrcstring{'127.0.0.1'} dbname=\pysrcstring{'database'} user=\pysrcstring{'postgres'}
\pysrcprompt{{\textgreater}{\textgreater}{\textgreater} }                password=\pysrcstring{'password'}")\end{alltt}
\begin{alltt}
\pysrcprompt{{\textgreater}{\textgreater}{\textgreater} }\pysrccomment{\# Read a network}
\pysrcprompt{{\textgreater}{\textgreater}{\textgreater} }\pysrccomment{\# Note 'my\_network' is the name of the network stored in the 'Graphs' table}
\pysrcprompt{{\textgreater}{\textgreater}{\textgreater} }network = nx\_pgnet.read(conn).pgnet(\pysrcstring{'my\_network'})\end{alltt}
Writing a NetworkX graph instance to a PostGIS schema:

Write the network to the same database but under a different name. 'EPSG' 
is the EPSE code for the output network geometry. Note if 'overwrite=True' 
then an existing network in the database of the same name will be 
overwritten.

\begin{alltt}
\pysrcprompt{{\textgreater}{\textgreater}{\textgreater} }epsg = 27700
\pysrcprompt{{\textgreater}{\textgreater}{\textgreater} }nx\_pgnet.write(conn).pgnet(network, \pysrcstring{'new\_network'}, epsg, overwrite=False)\end{alltt}
\textbf{Dependencies}

Python 2.6 or later NetworkX 1.6 or later OGR 1.8.0 or later

\textbf{Copyright (C)}

Tomas Holderness \& Newcastle University

Developed by Tom Holderness at Newcastle University School of Civil 
Engineering and Geosciences, geoinfomatics group:

David Alderson, Alistair Ford, Stuart Barr, Craig Robson.

\textbf{License}

This software is released under a BSD style license. See LICENSE.TXT or 
type nx\_pgnet.license() for full details.

\textbf{Credits}

Tomas Holderness, David Alderson, Alistair Ford, Stuart Barr and Craig 
Robson.

\textbf{Contact}

tom.holderness@ncl.ac.uk www.staff.ncl.ac.uk/tom.holderness

\textbf{Development Notes}

Where possible the PEP8/PEP257 style guide has been implemented.

To do:

\begin{enumerate}

\setlength{\parskip}{0.5ex}
  \item Check attribution of nodes from schema and non-schema sources (blank 
    old id fields are being copied over).

  \item Error  / warnings module.

  \item Investigate bug: "Warning 1: Geometry to be inserted is of type Line 
    String, whereas the layer geometry type is Multi Line String. Insertion
    is likely to fail!"

  \item Multi and directed graph support.

  \item 3D geometry support.

\end{enumerate}

\textbf{Version:} 0.9.2



\textbf{Author:} Tomas Holderness (tom.holderness@ncl.ac.uk)
David Alderson (david.alderson@ncl.ac.uk)
Alistair Ford (a.c.ford@ncl.ac.uk)
Stuart Barr (stuart.barr@ncl.ac.uk)



\textbf{License:} BSD style. See LICENSE.TXT




%%%%%%%%%%%%%%%%%%%%%%%%%%%%%%%%%%%%%%%%%%%%%%%%%%%%%%%%%%%%%%%%%%%%%%%%%%%
%%                               Variables                               %%
%%%%%%%%%%%%%%%%%%%%%%%%%%%%%%%%%%%%%%%%%%%%%%%%%%%%%%%%%%%%%%%%%%%%%%%%%%%

  \subsection{Variables}

    \vspace{-1cm}
\hspace{\varindent}\begin{longtable}{|p{\varnamewidth}|p{\vardescrwidth}|l}
\cline{1-2}
\cline{1-2} \centering \textbf{Name} & \centering \textbf{Description}& \\
\cline{1-2}
\endhead\cline{1-2}\multicolumn{3}{r}{\small\textit{continued on next page}}\\\endfoot\cline{1-2}
\endlastfoot\raggedright \_\-\_\-c\-r\-e\-a\-t\-e\-d\-\_\-\_\- & \raggedright \textbf{Value:} 
{\tt \texttt{'}\texttt{January 2012}\texttt{'}}&\\
\cline{1-2}
\raggedright \_\-\_\-p\-a\-c\-k\-a\-g\-e\-\_\-\_\- & \raggedright \textbf{Value:} 
{\tt None}&\\
\cline{1-2}
\end{longtable}


%%%%%%%%%%%%%%%%%%%%%%%%%%%%%%%%%%%%%%%%%%%%%%%%%%%%%%%%%%%%%%%%%%%%%%%%%%%
%%                           Class Description                           %%
%%%%%%%%%%%%%%%%%%%%%%%%%%%%%%%%%%%%%%%%%%%%%%%%%%%%%%%%%%%%%%%%%%%%%%%%%%%

    \index{nx\_pgnet \textit{(module)}!nx\_pgnet.Error \textit{(class)}|(}
\subsection{Class Error}

    \label{nx_pgnet:Error}
\begin{tabular}{cccccccccc}
% Line for object, linespec=[False, False, False]
\multicolumn{2}{r}{\settowidth{\BCL}{object}\multirow{2}{\BCL}{object}}
&&
&&
&&
  \\\cline{3-3}
  &&\multicolumn{1}{c|}{}
&&
&&
&&
  \\
% Line for exceptions.BaseException, linespec=[False, False]
\multicolumn{4}{r}{\settowidth{\BCL}{exceptions.BaseException}\multirow{2}{\BCL}{exceptions.BaseException}}
&&
&&
  \\\cline{5-5}
  &&&&\multicolumn{1}{c|}{}
&&
&&
  \\
% Line for exceptions.Exception, linespec=[False]
\multicolumn{6}{r}{\settowidth{\BCL}{exceptions.Exception}\multirow{2}{\BCL}{exceptions.Exception}}
&&
  \\\cline{7-7}
  &&&&&&\multicolumn{1}{c|}{}
&&
  \\
&&&&&&\multicolumn{2}{l}{\textbf{nx\_pgnet.Error}}
\end{tabular}

Class to handle network IO errors.


%%%%%%%%%%%%%%%%%%%%%%%%%%%%%%%%%%%%%%%%%%%%%%%%%%%%%%%%%%%%%%%%%%%%%%%%%%%
%%                                Methods                                %%
%%%%%%%%%%%%%%%%%%%%%%%%%%%%%%%%%%%%%%%%%%%%%%%%%%%%%%%%%%%%%%%%%%%%%%%%%%%

  \subsubsection{Methods}

    \vspace{0.5ex}

\hspace{.8\funcindent}\begin{boxedminipage}{\funcwidth}

    \raggedright \textbf{\_\_init\_\_}(\textit{self}, \textit{value})

\setlength{\parskip}{2ex}
    x.\_\_init\_\_(...) initializes x; see x.\_\_class\_\_.\_\_doc\_\_ for 
    signature

\setlength{\parskip}{1ex}
      Overrides: object.\_\_init\_\_ 	extit{(inherited documentation)}

    \end{boxedminipage}

    \vspace{0.5ex}

\hspace{.8\funcindent}\begin{boxedminipage}{\funcwidth}

    \raggedright \textbf{\_\_str\_\_}(\textit{self})

\setlength{\parskip}{2ex}
    str(x)

\setlength{\parskip}{1ex}
      Overrides: object.\_\_str\_\_ 	extit{(inherited documentation)}

    \end{boxedminipage}


\large{\textbf{\textit{Inherited from exceptions.Exception}}}

\begin{quote}
\_\_new\_\_()
\end{quote}

\large{\textbf{\textit{Inherited from exceptions.BaseException}}}

\begin{quote}
\_\_delattr\_\_(), \_\_getattribute\_\_(), \_\_getitem\_\_(), \_\_getslice\_\_(), \_\_reduce\_\_(), \_\_repr\_\_(), \_\_setattr\_\_(), \_\_setstate\_\_(), \_\_unicode\_\_()
\end{quote}

\large{\textbf{\textit{Inherited from object}}}

\begin{quote}
\_\_format\_\_(), \_\_hash\_\_(), \_\_reduce\_ex\_\_(), \_\_sizeof\_\_(), \_\_subclasshook\_\_()
\end{quote}

%%%%%%%%%%%%%%%%%%%%%%%%%%%%%%%%%%%%%%%%%%%%%%%%%%%%%%%%%%%%%%%%%%%%%%%%%%%
%%                              Properties                               %%
%%%%%%%%%%%%%%%%%%%%%%%%%%%%%%%%%%%%%%%%%%%%%%%%%%%%%%%%%%%%%%%%%%%%%%%%%%%

  \subsubsection{Properties}

    \vspace{-1cm}
\hspace{\varindent}\begin{longtable}{|p{\varnamewidth}|p{\vardescrwidth}|l}
\cline{1-2}
\cline{1-2} \centering \textbf{Name} & \centering \textbf{Description}& \\
\cline{1-2}
\endhead\cline{1-2}\multicolumn{3}{r}{\small\textit{continued on next page}}\\\endfoot\cline{1-2}
\endlastfoot\multicolumn{2}{|l|}{\textit{Inherited from exceptions.BaseException}}\\
\multicolumn{2}{|p{\varwidth}|}{\raggedright args, message}\\
\cline{1-2}
\multicolumn{2}{|l|}{\textit{Inherited from object}}\\
\multicolumn{2}{|p{\varwidth}|}{\raggedright \_\_class\_\_}\\
\cline{1-2}
\end{longtable}

    \index{nx\_pgnet \textit{(module)}!nx\_pgnet.Error \textit{(class)}|)}

%%%%%%%%%%%%%%%%%%%%%%%%%%%%%%%%%%%%%%%%%%%%%%%%%%%%%%%%%%%%%%%%%%%%%%%%%%%
%%                           Class Description                           %%
%%%%%%%%%%%%%%%%%%%%%%%%%%%%%%%%%%%%%%%%%%%%%%%%%%%%%%%%%%%%%%%%%%%%%%%%%%%

    \index{nx\_pgnet \textit{(module)}!nx\_pgnet.nisql \textit{(class)}|(}
\subsection{Class nisql}

    \label{nx_pgnet:nisql}
Contains wrappers for PostGIS network schema functions.

Where possible avoid using this class directly. Uses the read and write 
classes instead.


%%%%%%%%%%%%%%%%%%%%%%%%%%%%%%%%%%%%%%%%%%%%%%%%%%%%%%%%%%%%%%%%%%%%%%%%%%%
%%                                Methods                                %%
%%%%%%%%%%%%%%%%%%%%%%%%%%%%%%%%%%%%%%%%%%%%%%%%%%%%%%%%%%%%%%%%%%%%%%%%%%%

  \subsubsection{Methods}

    \label{nx_pgnet:nisql:__init__}
    \index{nx\_pgnet \textit{(module)}!nx\_pgnet.nisql \textit{(class)}!nx\_pgnet.nisql.\_\_init\_\_ \textit{(method)}}

    \vspace{0.5ex}

\hspace{.8\funcindent}\begin{boxedminipage}{\funcwidth}

    \raggedright \textbf{\_\_init\_\_}(\textit{self}, \textit{db\_conn})

    \vspace{-1.5ex}

    \rule{\textwidth}{0.5\fboxrule}
\setlength{\parskip}{2ex}
    Setup connection to be inherited by methods.

\setlength{\parskip}{1ex}
    \end{boxedminipage}

    \label{nx_pgnet:nisql:sql_function_check}
    \index{nx\_pgnet \textit{(module)}!nx\_pgnet.nisql \textit{(class)}!nx\_pgnet.nisql.sql\_function\_check \textit{(method)}}

    \vspace{0.5ex}

\hspace{.8\funcindent}\begin{boxedminipage}{\funcwidth}

    \raggedright \textbf{sql\_function\_check}(\textit{self}, \textit{function\_name})

    \vspace{-1.5ex}

    \rule{\textwidth}{0.5\fboxrule}
\setlength{\parskip}{2ex}
    Checks Postgres database for existence of specified function, if not 
    found raises error.

\setlength{\parskip}{1ex}
    \end{boxedminipage}

    \label{nx_pgnet:nisql:create_network_tables}
    \index{nx\_pgnet \textit{(module)}!nx\_pgnet.nisql \textit{(class)}!nx\_pgnet.nisql.create\_network\_tables \textit{(method)}}

    \vspace{0.5ex}

\hspace{.8\funcindent}\begin{boxedminipage}{\funcwidth}

    \raggedright \textbf{create\_network\_tables}(\textit{self}, \textit{prefix}, \textit{epsg}, \textit{directed}, \textit{multigraph})

    \vspace{-1.5ex}

    \rule{\textwidth}{0.5\fboxrule}
\setlength{\parskip}{2ex}
    Wrapper for ni\_create\_network\_tables function.

    Creates empty network schema PostGIS tables. Requires graph 'prefix 
    'name and srid to create empty network schema PostGIS tables.

    Returns True if succesful.

\setlength{\parskip}{1ex}
    \end{boxedminipage}

    \label{nx_pgnet:nisql:create_node_view}
    \index{nx\_pgnet \textit{(module)}!nx\_pgnet.nisql \textit{(class)}!nx\_pgnet.nisql.create\_node\_view \textit{(method)}}

    \vspace{0.5ex}

\hspace{.8\funcindent}\begin{boxedminipage}{\funcwidth}

    \raggedright \textbf{create\_node\_view}(\textit{self}, \textit{prefix})

    \vspace{-1.5ex}

    \rule{\textwidth}{0.5\fboxrule}
\setlength{\parskip}{2ex}
    Wrapper for ni\_create\_node\_view function.

    Creates a view containing node attributes and geometry values including
    int primary key suitable for QGIS.

    Requires network name ('prefix').

    Returns view name if succesful.

\setlength{\parskip}{1ex}
    \end{boxedminipage}

    \label{nx_pgnet:nisql:create_edge_view}
    \index{nx\_pgnet \textit{(module)}!nx\_pgnet.nisql \textit{(class)}!nx\_pgnet.nisql.create\_edge\_view \textit{(method)}}

    \vspace{0.5ex}

\hspace{.8\funcindent}\begin{boxedminipage}{\funcwidth}

    \raggedright \textbf{create\_edge\_view}(\textit{self}, \textit{prefix})

    \vspace{-1.5ex}

    \rule{\textwidth}{0.5\fboxrule}
\setlength{\parskip}{2ex}
    Wrapper for ni\_create\_edge\_view function.

    Creates a view containing edge attributes and edge geometry values. 
    Requires network name ('prefix').

    Returns view name if succesful.

\setlength{\parskip}{1ex}
    \end{boxedminipage}

    \label{nx_pgnet:nisql:add_graph_record}
    \index{nx\_pgnet \textit{(module)}!nx\_pgnet.nisql \textit{(class)}!nx\_pgnet.nisql.add\_graph\_record \textit{(method)}}

    \vspace{0.5ex}

\hspace{.8\funcindent}\begin{boxedminipage}{\funcwidth}

    \raggedright \textbf{add\_graph\_record}(\textit{self}, \textit{prefix}, \textit{directed}={\tt False}, \textit{multipath}={\tt False})

    \vspace{-1.5ex}

    \rule{\textwidth}{0.5\fboxrule}
\setlength{\parskip}{2ex}
    Wrapper for ni\_add\_graph\_record function.

    Creates a record in the Graphs table based on graph attributes.

    Returns new graph id of succesful.

\setlength{\parskip}{1ex}
    \end{boxedminipage}

    \label{nx_pgnet:nisql:node_geometry_equaility_check}
    \index{nx\_pgnet \textit{(module)}!nx\_pgnet.nisql \textit{(class)}!nx\_pgnet.nisql.node\_geometry\_equaility\_check \textit{(method)}}

    \vspace{0.5ex}

\hspace{.8\funcindent}\begin{boxedminipage}{\funcwidth}

    \raggedright \textbf{node\_geometry\_equaility\_check}(\textit{self}, \textit{prefix}, \textit{wkt}, \textit{srs})

    \vspace{-1.5ex}

    \rule{\textwidth}{0.5\fboxrule}
\setlength{\parskip}{2ex}
    Wrapper for ni\_node\_geometry\_equality\_check function.

    Checks if geometry already eixsts in nodes table.

    If not, returns None

\setlength{\parskip}{1ex}
    \end{boxedminipage}

    \label{nx_pgnet:nisql:edge_geometry_equaility_check}
    \index{nx\_pgnet \textit{(module)}!nx\_pgnet.nisql \textit{(class)}!nx\_pgnet.nisql.edge\_geometry\_equaility\_check \textit{(method)}}

    \vspace{0.5ex}

\hspace{.8\funcindent}\begin{boxedminipage}{\funcwidth}

    \raggedright \textbf{edge\_geometry\_equaility\_check}(\textit{self}, \textit{prefix}, \textit{wkt}, \textit{srs})

    \vspace{-1.5ex}

    \rule{\textwidth}{0.5\fboxrule}
\setlength{\parskip}{2ex}
    Wrapper for ni\_edge\_geometry\_equality\_check function.

    Checks if geometry already eixsts in nodes table.

    If not, return None

\setlength{\parskip}{1ex}
    \end{boxedminipage}

    \label{nx_pgnet:nisql:delete_network}
    \index{nx\_pgnet \textit{(module)}!nx\_pgnet.nisql \textit{(class)}!nx\_pgnet.nisql.delete\_network \textit{(method)}}

    \vspace{0.5ex}

\hspace{.8\funcindent}\begin{boxedminipage}{\funcwidth}

    \raggedright \textbf{delete\_network}(\textit{self}, \textit{prefix})

    \vspace{-1.5ex}

    \rule{\textwidth}{0.5\fboxrule}
\setlength{\parskip}{2ex}
    Wrapper for ni\_delete\_network function.

    Deletes a network entry from the Graphs table and drops associated 
    tables.

\setlength{\parskip}{1ex}
    \end{boxedminipage}

    \index{nx\_pgnet \textit{(module)}!nx\_pgnet.nisql \textit{(class)}|)}

%%%%%%%%%%%%%%%%%%%%%%%%%%%%%%%%%%%%%%%%%%%%%%%%%%%%%%%%%%%%%%%%%%%%%%%%%%%
%%                           Class Description                           %%
%%%%%%%%%%%%%%%%%%%%%%%%%%%%%%%%%%%%%%%%%%%%%%%%%%%%%%%%%%%%%%%%%%%%%%%%%%%

    \index{nx\_pgnet \textit{(module)}!nx\_pgnet.read \textit{(class)}|(}
\subsection{Class read}

    \label{nx_pgnet:read}
Class to read and build networks from PostGIS schema network tables.


%%%%%%%%%%%%%%%%%%%%%%%%%%%%%%%%%%%%%%%%%%%%%%%%%%%%%%%%%%%%%%%%%%%%%%%%%%%
%%                                Methods                                %%
%%%%%%%%%%%%%%%%%%%%%%%%%%%%%%%%%%%%%%%%%%%%%%%%%%%%%%%%%%%%%%%%%%%%%%%%%%%

  \subsubsection{Methods}

    \label{nx_pgnet:read:__init__}
    \index{nx\_pgnet \textit{(module)}!nx\_pgnet.read \textit{(class)}!nx\_pgnet.read.\_\_init\_\_ \textit{(method)}}

    \vspace{0.5ex}

\hspace{.8\funcindent}\begin{boxedminipage}{\funcwidth}

    \raggedright \textbf{\_\_init\_\_}(\textit{self}, \textit{db\_conn})

    \vspace{-1.5ex}

    \rule{\textwidth}{0.5\fboxrule}
\setlength{\parskip}{2ex}
    Setup connection to be inherited by methods.

\setlength{\parskip}{1ex}
    \end{boxedminipage}

    \label{nx_pgnet:read:getfieldinfo}
    \index{nx\_pgnet \textit{(module)}!nx\_pgnet.read \textit{(class)}!nx\_pgnet.read.getfieldinfo \textit{(method)}}

    \vspace{0.5ex}

\hspace{.8\funcindent}\begin{boxedminipage}{\funcwidth}

    \raggedright \textbf{getfieldinfo}(\textit{self}, \textit{lyr}, \textit{feature}, \textit{flds})

    \vspace{-1.5ex}

    \rule{\textwidth}{0.5\fboxrule}
\setlength{\parskip}{2ex}
    Get information about fields from a table (as OGR feature).

\setlength{\parskip}{1ex}
    \end{boxedminipage}

    \label{nx_pgnet:read:pgnet_edges}
    \index{nx\_pgnet \textit{(module)}!nx\_pgnet.read \textit{(class)}!nx\_pgnet.read.pgnet\_edges \textit{(method)}}

    \vspace{0.5ex}

\hspace{.8\funcindent}\begin{boxedminipage}{\funcwidth}

    \raggedright \textbf{pgnet\_edges}(\textit{self}, \textit{graph})

    \vspace{-1.5ex}

    \rule{\textwidth}{0.5\fboxrule}
\setlength{\parskip}{2ex}
    Reads edges from edge and edge\_geometry tables and add to graph.

\setlength{\parskip}{1ex}
    \end{boxedminipage}

    \label{nx_pgnet:read:pgnet_nodes}
    \index{nx\_pgnet \textit{(module)}!nx\_pgnet.read \textit{(class)}!nx\_pgnet.read.pgnet\_nodes \textit{(method)}}

    \vspace{0.5ex}

\hspace{.8\funcindent}\begin{boxedminipage}{\funcwidth}

    \raggedright \textbf{pgnet\_nodes}(\textit{self}, \textit{graph})

    \vspace{-1.5ex}

    \rule{\textwidth}{0.5\fboxrule}
\setlength{\parskip}{2ex}
    Reads nodes from node table and add to graph.

\setlength{\parskip}{1ex}
    \end{boxedminipage}

    \label{nx_pgnet:read:graph_table}
    \index{nx\_pgnet \textit{(module)}!nx\_pgnet.read \textit{(class)}!nx\_pgnet.read.graph\_table \textit{(method)}}

    \vspace{0.5ex}

\hspace{.8\funcindent}\begin{boxedminipage}{\funcwidth}

    \raggedright \textbf{graph\_table}(\textit{self}, \textit{prefix})

    \vspace{-1.5ex}

    \rule{\textwidth}{0.5\fboxrule}
\setlength{\parskip}{2ex}
    Reads the attributes of a graph from the graph table.

    Returns attributes as a dict of variables.

\setlength{\parskip}{1ex}
    \end{boxedminipage}

    \label{nx_pgnet:read:pgnet}
    \index{nx\_pgnet \textit{(module)}!nx\_pgnet.read \textit{(class)}!nx\_pgnet.read.pgnet \textit{(method)}}

    \vspace{0.5ex}

\hspace{.8\funcindent}\begin{boxedminipage}{\funcwidth}

    \raggedright \textbf{pgnet}(\textit{self}, \textit{prefix})

    \vspace{-1.5ex}

    \rule{\textwidth}{0.5\fboxrule}
\setlength{\parskip}{2ex}
    Read a network from PostGIS network schema tables.

    Returns instance of networkx.Graph().

\setlength{\parskip}{1ex}
    \end{boxedminipage}

    \index{nx\_pgnet \textit{(module)}!nx\_pgnet.read \textit{(class)}|)}

%%%%%%%%%%%%%%%%%%%%%%%%%%%%%%%%%%%%%%%%%%%%%%%%%%%%%%%%%%%%%%%%%%%%%%%%%%%
%%                           Class Description                           %%
%%%%%%%%%%%%%%%%%%%%%%%%%%%%%%%%%%%%%%%%%%%%%%%%%%%%%%%%%%%%%%%%%%%%%%%%%%%

    \index{nx\_pgnet \textit{(module)}!nx\_pgnet.write \textit{(class)}|(}
\subsection{Class write}

    \label{nx_pgnet:write}
Class to write NetworkX instance to PostGIS network schema tables.


%%%%%%%%%%%%%%%%%%%%%%%%%%%%%%%%%%%%%%%%%%%%%%%%%%%%%%%%%%%%%%%%%%%%%%%%%%%
%%                                Methods                                %%
%%%%%%%%%%%%%%%%%%%%%%%%%%%%%%%%%%%%%%%%%%%%%%%%%%%%%%%%%%%%%%%%%%%%%%%%%%%

  \subsubsection{Methods}

    \label{nx_pgnet:write:__init__}
    \index{nx\_pgnet \textit{(module)}!nx\_pgnet.write \textit{(class)}!nx\_pgnet.write.\_\_init\_\_ \textit{(method)}}

    \vspace{0.5ex}

\hspace{.8\funcindent}\begin{boxedminipage}{\funcwidth}

    \raggedright \textbf{\_\_init\_\_}(\textit{self}, \textit{db\_conn})

    \vspace{-1.5ex}

    \rule{\textwidth}{0.5\fboxrule}
\setlength{\parskip}{2ex}
    Setup connection to be inherited by methods.

\setlength{\parskip}{1ex}
    \end{boxedminipage}

    \label{nx_pgnet:write:getlayer}
    \index{nx\_pgnet \textit{(module)}!nx\_pgnet.write \textit{(class)}!nx\_pgnet.write.getlayer \textit{(method)}}

    \vspace{0.5ex}

\hspace{.8\funcindent}\begin{boxedminipage}{\funcwidth}

    \raggedright \textbf{getlayer}(\textit{self}, \textit{tablename})

    \vspace{-1.5ex}

    \rule{\textwidth}{0.5\fboxrule}
\setlength{\parskip}{2ex}
    Get a PostGIS table by name and return as OGR layer.

    Else, return None.

\setlength{\parskip}{1ex}
    \end{boxedminipage}

    \label{nx_pgnet:write:netgeometry}
    \index{nx\_pgnet \textit{(module)}!nx\_pgnet.write \textit{(class)}!nx\_pgnet.write.netgeometry \textit{(method)}}

    \vspace{0.5ex}

\hspace{.8\funcindent}\begin{boxedminipage}{\funcwidth}

    \raggedright \textbf{netgeometry}(\textit{self}, \textit{key}, \textit{data})

    \vspace{-1.5ex}

    \rule{\textwidth}{0.5\fboxrule}
\setlength{\parskip}{2ex}
    Create OGR geometry from a NetworkX Graph using Wkb/Wkt attributes.

\setlength{\parskip}{1ex}
    \end{boxedminipage}

    \label{nx_pgnet:write:create_feature}
    \index{nx\_pgnet \textit{(module)}!nx\_pgnet.write \textit{(class)}!nx\_pgnet.write.create\_feature \textit{(method)}}

    \vspace{0.5ex}

\hspace{.8\funcindent}\begin{boxedminipage}{\funcwidth}

    \raggedright \textbf{create\_feature}(\textit{self}, \textit{lyr}, \textit{attributes}={\tt None}, \textit{geometry}={\tt None})

    \vspace{-1.5ex}

    \rule{\textwidth}{0.5\fboxrule}
\setlength{\parskip}{2ex}
    Wrapper for OGR CreateFeature function.

    Creates a feature in the specified table with geometry and attributes.

\setlength{\parskip}{1ex}
    \end{boxedminipage}

    \label{nx_pgnet:write:create_attribute_map}
    \index{nx\_pgnet \textit{(module)}!nx\_pgnet.write \textit{(class)}!nx\_pgnet.write.create\_attribute\_map \textit{(method)}}

    \vspace{0.5ex}

\hspace{.8\funcindent}\begin{boxedminipage}{\funcwidth}

    \raggedright \textbf{create\_attribute\_map}(\textit{self}, \textit{lyr}, \textit{g\_obj}, \textit{fields})

    \vspace{-1.5ex}

    \rule{\textwidth}{0.5\fboxrule}
\setlength{\parskip}{2ex}
    Build a dict of attribute field names, data and OGR data types.

    Accepts graph object (either node or edge), fields and returns 
    attribute dictionary.

\setlength{\parskip}{1ex}
    \end{boxedminipage}

    \label{nx_pgnet:write:update_graph_table}
    \index{nx\_pgnet \textit{(module)}!nx\_pgnet.write \textit{(class)}!nx\_pgnet.write.update\_graph\_table \textit{(method)}}

    \vspace{0.5ex}

\hspace{.8\funcindent}\begin{boxedminipage}{\funcwidth}

    \raggedright \textbf{update\_graph\_table}(\textit{self}, \textit{graph})

    \vspace{-1.5ex}

    \rule{\textwidth}{0.5\fboxrule}
\setlength{\parskip}{2ex}
    Update the Graph table and return newly assigned Graph ID.

\setlength{\parskip}{1ex}
    \end{boxedminipage}

    \label{nx_pgnet:write:pgnet_edge}
    \index{nx\_pgnet \textit{(module)}!nx\_pgnet.write \textit{(class)}!nx\_pgnet.write.pgnet\_edge \textit{(method)}}

    \vspace{0.5ex}

\hspace{.8\funcindent}\begin{boxedminipage}{\funcwidth}

    \raggedright \textbf{pgnet\_edge}(\textit{self}, \textit{edge\_attributes}, \textit{edge\_geom})

    \vspace{-1.5ex}

    \rule{\textwidth}{0.5\fboxrule}
\setlength{\parskip}{2ex}
    Write an edge to Edge and Edge\_Geometry tables.

\setlength{\parskip}{1ex}
    \end{boxedminipage}

    \label{nx_pgnet:write:pgnet_node}
    \index{nx\_pgnet \textit{(module)}!nx\_pgnet.write \textit{(class)}!nx\_pgnet.write.pgnet\_node \textit{(method)}}

    \vspace{0.5ex}

\hspace{.8\funcindent}\begin{boxedminipage}{\funcwidth}

    \raggedright \textbf{pgnet\_node}(\textit{self}, \textit{node\_attributes}, \textit{node\_geom})

    \vspace{-1.5ex}

    \rule{\textwidth}{0.5\fboxrule}
\setlength{\parskip}{2ex}
    Write a node to a Node table.

    Return the newly assigned NodeID.

\setlength{\parskip}{1ex}
    \end{boxedminipage}

    \label{nx_pgnet:write:pgnet}
    \index{nx\_pgnet \textit{(module)}!nx\_pgnet.write \textit{(class)}!nx\_pgnet.write.pgnet \textit{(method)}}

    \vspace{0.5ex}

\hspace{.8\funcindent}\begin{boxedminipage}{\funcwidth}

    \raggedright \textbf{pgnet}(\textit{self}, \textit{network}, \textit{tablename\_prefix}, \textit{srs}, \textit{overwrite}={\tt False}, \textit{directed}={\tt False}, \textit{multigraph}={\tt False})

    \vspace{-1.5ex}

    \rule{\textwidth}{0.5\fboxrule}
\setlength{\parskip}{2ex}
    Write NetworkX instance to PostGIS network schema tables.

    Updates Graph table with new network.

    Note that schema constrains must be applied in database. There are no 
    checks for database errors here.

\setlength{\parskip}{1ex}
    \end{boxedminipage}

    \index{nx\_pgnet \textit{(module)}!nx\_pgnet.write \textit{(class)}|)}
    \index{nx\_pgnet \textit{(module)}|)}
