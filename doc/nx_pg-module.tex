%
% API Documentation for NetworkX - PostGIS Python interface
% Module nx_pg
%
% Generated by epydoc 3.0.1
% [Wed Feb 15 10:34:11 2012]
%

%%%%%%%%%%%%%%%%%%%%%%%%%%%%%%%%%%%%%%%%%%%%%%%%%%%%%%%%%%%%%%%%%%%%%%%%%%%
%%                          Module Description                           %%
%%%%%%%%%%%%%%%%%%%%%%%%%%%%%%%%%%%%%%%%%%%%%%%%%%%%%%%%%%%%%%%%%%%%%%%%%%%

    \index{nx\_pg \textit{(module)}|(}
\section{Module nx\_pg}

    \label{nx_pg}
\begin{alltt}

------------
NetworkX is a python library for graph analysis. Using edge and node 
attribution it can be used for spatial network analysis (with geography stored 
as node/edge attributes). This module supports the use of NetworkX for the
development of network analysis of spatial networks stored in a PostGIS 
spatial database by acting as an interface to node and edge tables which 
contain graph nodes and edges.

-----
Notes
-----
Node support
Note that nodes are defined by the edges at network creation and the reading 
of node tables independently is not currently supported (because without 
defining a primary key/foreign key relationship this can easily break the :
network). This issue is solved in nx\_pgnet which should be used for proper
storage of networks in PostGIS tables. To read/write PostGIS networks 
(as defined by a network schema use) the nx\_pgnet module.

Output tables
For each network written two tables are created: edges and nodes. 
This representation is similar to that of the nx\_shp module 
(for reading/writing network shapefiles). 

Coordinate system support
nx\_pg has no support for defining a coordinate system of the output tables. 
Geometry is written without an SRS value, when viewing using a GIS you must
specify the correct coordinate system for the network. nx\_pgnet has coordinate
systems support for network tables.

Graph/Network terms
Note that the terms 'graph' and 'network' are used interchangeably within the 
software and documentation. To some extent a 'graph' refers to a topological 
object (often in memory) with none or limited attribution, whilst a 'network' 
refers to a graph object with attribution of edges and nodes and with 
geography defined, although this is not always the case.

----------------
Module structure    
----------------
The module has two key functions:
    - read\_pg:
        Function to create NetworkX graph instance from PostGIS table(s) 
        representing edge and node objects.
    - write\_pg:
        Function to create PostGIS tables (edge and node) from NetworkX graph
        instance.
        
    - Other functions support the read and write operations.
    
--------------------
Database connections
--------------------
Connections to PostGIS are created using the OGR simple feature library and are
passed to the read() and write() classes. See http://www.gdal.ogr/ogr

Connections are mutually exclusive between read\_pg and write\_pg, although you 
can of course read and write to the same database. 
You must pass a valid connection to the read or write classes for the module 
to work.

To create a connection using the OGR Python (SWIG) OGR bindings to a database
on localhost:
    
    import osgeo.ogr as ogr
    conn = ogr.Open("PG: host='127.0.0.1' dbname='database' user='postgres'
                    password='password'")
    
--------
Examples
--------
The following are examples of high level read and write network operations. For
more detailed information see method documentation below.

Reading a network from PostGIS table of LINESTRINGS representing edges:
    
    import nx\_pg
    import osgeo.ogr as ogr
    
    \# Create a connection
    conn = ogr.Open("PG: host='127.0.0.1' dbname='database' user='postgres'
                    password='password'")

    \# Read a network
    \# Note 'my\_network' is the name of the network stored in the 'Graphs' table
    
    network = nx\_pg.read\_pg(conn, 'tablename')
    

Writing a NetworkX graph instance to a PostGIS schema:
    
    \# Write the network to the same database but under a different name.
    \# Note if 'overwrite=True' then an existing network in the database of the 
    \# same name will be overwritten.
    
    nx\_pg.write\_pg(conn, network, 'new\_network, overwrite=False')
    
------------
Dependencies
------------
Python 2.6 or later
NetworkX 1.6 or later
OGR 1.8.0 or later

-------------
Copyright (C)
-------------
Tomas Holderness / Newcastle University

Developed by Tom Holderness at Newcastle University School of Civil Engineering
and Geosciences, geoinfomatics group:

---------------
Acknowledgement 
---------------
Acknowledgement must be made to the nx\_shp developers as much of the 
functionality of this module is the same.

-------
License
-------
This software is released under a BSD style license which must be compatible
with the NetworkX license because of similarities with NetworkX source code.:
    
See LICENSE.TXT or type
nx\_pg.license() for full details.

-------
Credits
-------
Tomas Holderness, David Alderson, Alistair Ford, Stuart Barr and Craig Robson.

-------
Contact
-------
tom.holderness@ncl.ac.uk
www.staff.ncl.ac.uk/tom.holderness
\end{alltt}

\textbf{Version:} 0.9.1



\textbf{Author:} Tom Holderness




%%%%%%%%%%%%%%%%%%%%%%%%%%%%%%%%%%%%%%%%%%%%%%%%%%%%%%%%%%%%%%%%%%%%%%%%%%%
%%                               Functions                               %%
%%%%%%%%%%%%%%%%%%%%%%%%%%%%%%%%%%%%%%%%%%%%%%%%%%%%%%%%%%%%%%%%%%%%%%%%%%%

  \subsection{Functions}

    \label{nx_pg:getfieldinfo}
    \index{nx\_pg \textit{(module)}!nx\_pg.getfieldinfo \textit{(function)}}

    \vspace{0.5ex}

\hspace{.8\funcindent}\begin{boxedminipage}{\funcwidth}

    \raggedright \textbf{getfieldinfo}(\textit{lyr}, \textit{feature}, \textit{flds})

    \vspace{-1.5ex}

    \rule{\textwidth}{0.5\fboxrule}
\setlength{\parskip}{2ex}
    Get information about fields from a table (as OGR feature).

\setlength{\parskip}{1ex}
    \end{boxedminipage}

    \label{nx_pg:read_pg}
    \index{nx\_pg \textit{(module)}!nx\_pg.read\_pg \textit{(function)}}

    \vspace{0.5ex}

\hspace{.8\funcindent}\begin{boxedminipage}{\funcwidth}

    \raggedright \textbf{read\_pg}(\textit{conn}, \textit{tablename})

    \vspace{-1.5ex}

    \rule{\textwidth}{0.5\fboxrule}
\setlength{\parskip}{2ex}
    Read a network from PostGIS tables of line geometry.

    Returns instance of networkx.Graph().

\setlength{\parskip}{1ex}
    \end{boxedminipage}

    \label{nx_pg:netgeometry}
    \index{nx\_pg \textit{(module)}!nx\_pg.netgeometry \textit{(function)}}

    \vspace{0.5ex}

\hspace{.8\funcindent}\begin{boxedminipage}{\funcwidth}

    \raggedright \textbf{netgeometry}(\textit{key}, \textit{data})

    \vspace{-1.5ex}

    \rule{\textwidth}{0.5\fboxrule}
\setlength{\parskip}{2ex}
    Create OGR geometry from a NetworkX Graph using Wkb/Wkt attributes.

\setlength{\parskip}{1ex}
    \end{boxedminipage}

    \label{nx_pg:create_feature}
    \index{nx\_pg \textit{(module)}!nx\_pg.create\_feature \textit{(function)}}

    \vspace{0.5ex}

\hspace{.8\funcindent}\begin{boxedminipage}{\funcwidth}

    \raggedright \textbf{create\_feature}(\textit{geometry}, \textit{lyr}, \textit{attributes}={\tt None})

    \vspace{-1.5ex}

    \rule{\textwidth}{0.5\fboxrule}
\setlength{\parskip}{2ex}
    Wrapper for OGR CreateFeature function.

    Creates a feature in the specified table with geometry and attributes.

\setlength{\parskip}{1ex}
    \end{boxedminipage}

    \label{nx_pg:write_pg}
    \index{nx\_pg \textit{(module)}!nx\_pg.write\_pg \textit{(function)}}

    \vspace{0.5ex}

\hspace{.8\funcindent}\begin{boxedminipage}{\funcwidth}

    \raggedright \textbf{write\_pg}(\textit{conn}, \textit{network}, \textit{tablename\_prefix}, \textit{overwrite}={\tt False})

    \vspace{-1.5ex}

    \rule{\textwidth}{0.5\fboxrule}
\setlength{\parskip}{2ex}
    Write NetworkX instance to PostGIS edge and node tables.

\setlength{\parskip}{1ex}
    \end{boxedminipage}


%%%%%%%%%%%%%%%%%%%%%%%%%%%%%%%%%%%%%%%%%%%%%%%%%%%%%%%%%%%%%%%%%%%%%%%%%%%
%%                               Variables                               %%
%%%%%%%%%%%%%%%%%%%%%%%%%%%%%%%%%%%%%%%%%%%%%%%%%%%%%%%%%%%%%%%%%%%%%%%%%%%

  \subsection{Variables}

    \vspace{-1cm}
\hspace{\varindent}\begin{longtable}{|p{\varnamewidth}|p{\vardescrwidth}|l}
\cline{1-2}
\cline{1-2} \centering \textbf{Name} & \centering \textbf{Description}& \\
\cline{1-2}
\endhead\cline{1-2}\multicolumn{3}{r}{\small\textit{continued on next page}}\\\endfoot\cline{1-2}
\endlastfoot\raggedright \_\-\_\-c\-r\-e\-a\-t\-e\-d\-\_\-\_\- & \raggedright \textbf{Value:} 
{\tt \texttt{'}\texttt{Thu Jan 19 15:55:13 2012}\texttt{'}}&\\
\cline{1-2}
\raggedright \_\-\_\-y\-e\-a\-r\-\_\-\_\- & \raggedright \textbf{Value:} 
{\tt \texttt{'}\texttt{2011}\texttt{'}}&\\
\cline{1-2}
\raggedright \_\-\_\-p\-a\-c\-k\-a\-g\-e\-\_\-\_\- & \raggedright \textbf{Value:} 
{\tt None}&\\
\cline{1-2}
\end{longtable}

    \index{nx\_pg \textit{(module)}|)}
